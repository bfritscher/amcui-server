\documentclass[a4paper]{article}

\usepackage[utf8x]{inputenc}    
\usepackage[T1]{fontenc}
\usepackage[francais,lang=FR]{automultiplechoice}
%%% completemulti, ensemble
\usepackage{fp}
\usepackage{array}
\usepackage{tabularx}
\usepackage{multirow}
\usepackage{multicol}
\usepackage{enumitem}
\usepackage{courier} 
\usepackage{listings}
\usepackage{color}

\usepackage{tikz} %for oval
\usepackage[scaled]{helvet}

\definecolor{lightgray}{rgb}{.9,.9,.9}
\definecolor{darkgray}{rgb}{.2,.2,.2}
\definecolor{purple}{rgb}{0.65, 0.12, 0.82}

\renewcommand{\familydefault}{\sfdefault}

%%% fix Openquestion bug until version 1.3
\def\AMCOpen{\hfill\AMCformatChoices{\AMCopenShow}{\AMCopenHide}}

\lstdefinelanguage{JavaScript}{
  keywords={typeof, new, true, false, catch, function, return, null, catch, switch, var, if, in, while, for, do, else, case, break},
  keywordstyle=\color{blue}\bfseries,
  ndkeywords={class, export, boolean, throw, implements, import, this},
  ndkeywordstyle=\color{darkgray}\bfseries,
  identifierstyle=\color{black},
  sensitive=false,
  comment=[l]{//},
  morecomment=[s]{/*}{*/},
  commentstyle=\color{purple}\ttfamily,
  stringstyle=\color{darkgray}\ttfamily,
  morestring=[b]',
  morestring=[b]"
}

\lstset{
   language=JavaScript,
   %backgroundcolor=\color{lightgray},
   extendedchars=true,
   inputencoding=utf8x,
   literate={é}{{\'e}}1 {ê}{{\^{e}}}1 {è}{{\`{e}}}1 {à}{{\`{a}}}1,
   basicstyle=\footnotesize\ttfamily,
   showstringspaces=false,
   showspaces=false,
   numbers=left,
   numberstyle=\footnotesize,
   numbersep=9pt,
   tabsize=2,
   breaklines=true,
   showtabs=false,
   captionpos=b,
   escapeinside={(*@}{@*)}
}


%%% \geometry{hmargin=3cm,headheight=2cm,headsep=.3cm,footskip=1cm,top=3.5cm,bottom=2.5cm}

\begin{document}

\AMCrandomseed{1237893}
\AMCboxColor{red}
\AMCboxDimensions{shape=oval}


%%% Scoring strategy:
\scoringDefaultM{e=0,v=0,b=0,m=0,p=0,d=0,mz=1}




\def\AMCformQuestion#1{\vspace{\AMCformVSpace}\par {\sc Question #1 :} }   

%%% liste des questions:
\input{questions_definition.tex}

\exemplaire{1}{

%%% debut de l'en-tête des copies :    
\setlength{\parindent}{0pt}

\noindent{\textbf{
%%% Examen de fin de semestre
} \hfill \includegraphics[]{src/graphics/logo_heg.png}}

\vspace{2ex}

\noindent{\textbf{1PT/1TPart} \hfill \bf Session \ACMUIsession}

\ACMUImatiere  \hfill \ACMUIteacher

\vspace*{1cm}

{\renewcommand{\arraystretch}{4}
\begin{tabularx}{15cm}{p{6cm}p{0.5cm}p{7cm}}
\hline
\underline{Unité d'enseignement} & : & \textbf{\ACMUImatiere}    \\ \hline
\underline{Professeur-e/s} & : & \textbf{\ACMUIteacher}  \\ \hline
\underline{Moyens auxiliaires autorises} & : & \textbf{Toute documentation papier} \\ \hline
\underline{Date} & : & \textbf{11.06.2015} \\ \hline
\underline{Temps à disposition} & : & \textbf{90'} \\ \hline
\underline{Cotation utilisée pour la correction} & : & \textbf{Indiquée dans la donnée} \\ \hline
\underline{Consignes} & : & Écrire au stylo noir ou bleu. Bon travail \\ \hline
\underline{Nom et prénom de l'étudiant-e} & : & \\ \hline
\end{tabularx}
}


\vspace*{2cm}Visa du professeur: \dotfill
\newpage

%%% \vspace{3cm}
%%% 
%%% \begin{center}
%%% \textit{Cette page a été laissée intentionnellement blanche}
%%% \end{center}


%%%  \newpage
%\multicolumn{1}{|c|}{\bf 

%%% {
%%% \em\setlength{\parindent}{0pt}
%%% 
%%% Les questions faisant apparaître le symbole \multiSymbole{} peuvent
%%% présenter une ou plusieurs bonnes réponses. Les autres ont
%%%   une unique bonne réponse.
%%% \newline
%%% 
%%% A la fin de l'examen, vous rendrez cet énoncé et la feuille de réponses distribuée avec l'énoncé.
%%%  Les deux documents doivent porter votre nom et prénom.
%%% \newline
%%% 
%%% 
%%% }\hspace*{\fill}
\vspace{1ex}

%%% fin de l'en-tête




\AMCcode{etu}{2}
\hspace*{\fill}
\begin{minipage}[b]{6.5cm}
\setlength{\parindent}{0pt}
$\longleftarrow{}$\hspace{0pt plus 1cm} veuillez ne rien coder ci-contre,
et inscrire votre nom et prénom \textit{lisiblement} ci-dessous.

\vspace{3ex}

\champnom{
  \fbox{    
    \begin{minipage}{.9\linewidth}
      Nom:
      
      \vspace*{.5cm}\dotfill

      Prénom:         

      \vspace*{.5cm}\dotfill

      \vspace*{1mm}
    \end{minipage}
  }
}
  \hfill\vspace{5ex}
  \end{minipage}

  \vspace{1ex}
{\bf\em\setlength{\parindent}{0pt} Les réponses aux questions sont à donner \textit{exclusivement} sur cette feuille: les réponses données sur d'autres feuilles ne seront \textit{pas prises en compte}.


%%% Coloriez complètement et proprement chaque case choisie, afin d'obtenir le meilleur contraste possible
%%% par rapport à une case vide. Ne coloriez pas les cases au crayon à papier, 

N'utilisez pas de couleur rouge (le programme correcteur ne voit pas les couleurs). Utilisez plutôt du noir ou du bleu foncé.

%%% Pour corriger une éventuelle erreur de saisie, veuillez utiliser si possible un correcteur liquide (Tipp-Ex).

\vspace{1ex}
Pour les questions ouvertes NE PAS COCHER les cases de points!
}

\input{questions_layout.tex}


%%% \AMCdebutFormulaire    

%%% début de l'en-tête de la feuille de réponses
}

\end{document}